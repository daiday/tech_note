\documentclass[12pt]{article}
\usepackage{geometry}
\geometry{a4paper}


\usepackage{color}
\usepackage{hyperref}
\usepackage{amsmath}
\usepackage{amsfonts}
\usepackage{amssymb}
\usepackage{graphicx}
\usepackage{tcolorbox}
\usepackage{listings}
\usepackage{here}
\usepackage{txfonts}
\usepackage{algorithm}
\usepackage{algorithmic}
\usepackage{siunitx}
\usepackage{xcolor}
\usepackage{ascmac}
%\usepackage{fancybx}

\lstset {language = c++,
  basicstyle = \ttfamily \scriptsize,
  commentstyle = \textit,
  frame = tRBl,
  framesep = 5pt,
  showstringspaces = false,
  numbers = left,
  stepnumber = 1,
  numberstyle = \tiny,
  tabsize = 2,
  keywordstyle = \bfseries \color{blue},
  stringstyle=\color{magenta},
  commentstyle=\color{red},
  morecomment=[l][\color{red}]{\#}
  showstringspaces=false, % don't mark spaces in strings
}
\newcommand{\bi}[1]{\mathbf{#1}}
\newcommand{\bs}[1]{\boldsymbol{#1}}  % bold for greek characters
\newcommand{\bbR}{\mathbb{R}}

\author{Nobuyuki Umetani}
\title{Laplacian Mesh Deformation}
\author{Nobuyuki Umetani}

\begin{document}
\maketitle

\tableofcontents

\section{Linear Model}

Laplacian on the three-dimensional mesh 

Given a point $i$, the Laplacian of the mesh at the point is defined as:
%
\begin{equation}
    \bi{v}_i = \bi{x}_i - \sum_{j\in \cal{N}_i} \omega_{ij} \bi{x}_j,
\end{equation}
%
where $\omega_{ij}$ is a weight satisfies
%
\begin{equation}
    \sum_{j\in \cal{N}_i} \omega_{ij} = 1.
\end{equation}

Elastic energy is defined as the square sum of the difference of the Laplacian
\begin{eqnarray}
    E = \sum_{i\in N}E_i
\end{eqnarray}
where the $N$ is the set of points and the energy associated with the vertex $i$ is written as
\begin{eqnarray}
    E_i = \frac{1}{2}\left(\bi{x}_i - \sum_{j\in \cal{N}_i} \omega_{ij} \bi{x}_j\right)^T \left(\bi{x}_i - \sum_{k\in \cal{N}_i} \omega_{ik} \bi{x}_k\right)
\end{eqnarray}

 and $\bar{\bi{v}}_i$ is the Laplacian of the point $i$ in the undeformed configuration.

\begin{eqnarray}
    \frac{\partial E_i}{\partial \bi{x}_i} = 
\end{eqnarray}


\end{document}


